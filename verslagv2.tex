\title{Semantiek van WhIole}
\author{
          Tim van Dijk
          S4477073
        \and
          Nikki van der Gouw
          S4463412
        \and
          Robin Tonen 
          S4486668
}
\date{\today}

\documentclass[12pt]{article}
\usepackage[margin=1in]{geometry}
\usepackage{enumitem}
\usepackage{listings} 
\PassOptionsToPackage{hyphens}{url}\usepackage{hyperref}
\usepackage{graphicx}
\usepackage[dutch]{babel}
\usepackage[bottom]{footmisc}
\usepackage{listings}
\graphicspath{{images/}}
\usepackage{stmaryrd}
\usepackage[fleqn]{amsmath}
\usepackage{amssymb}
\usepackage{dsfont}
\usepackage{ stmaryrd }
\usepackage[utf8]{inputenc}
\usepackage[T1]{fontenc}
\usepackage{lmodern}

\graphicspath{{images/}}
\lstset{breaklines = true}
\newcommand{\RA}{\mathcal{RA}}
\newcommand{\RB}{\mathcal{RB}}
\newcommand{\RO}{\mathcal{RO}}
\newcommand{\n}{\newline}

\begin{document}
\maketitle

%TODO: loop tekst na op woordherhaling
\begin{abstract}
Dit werkstuk beschrijft de syntax en semantiek van WhIole, een door Io geinspireerde uitbreiding van While. Io is een puur objectge\"orienteerde programmeertaal. Hierbij wordt WhIole met natuurlijke semantiek beschreven. Vervolgens gebruiken we de beschreven syntax en semantiek om
de mogelijkheden van Io te laten zien.
\end{abstract}

%Zorg dat je een korte inleiding schrijft waarin je zo concreet mogelijk uitlegt wat je gaat doen.
%Beschrijf hierbij ook wat er zo bijzonder is aan het gekozen onderwerp ten opzichte van de standaardtaal While.
%Neem in de inleiding ook alvast een voorbeeldprogramma op waarin iets interessants gebeurt.
%Zo'n programma helpt om duidelijk te krijgen hoe programma's in jullie taal er uitzien.
\section{Introductie Io}
Io is in 2002 gemaakt door Steve Dekorte omdat hij zijn begrip over de werking van programmeertalen wilde verbreden.
Het bijzondere aan Io is dat het een heel minimalistische en puur objectgeoriënteerde programmeertaal is die gebaseerd is op prototypes. De taal is geïnspireerd door Smalltalk. Alles, inclusief objecten, bestaat uit `messages' die worden doorgegeven.
Dit houdt in dat er alleen maar objecten bestaan en dat overerving plaats vindt door bestaande objecten te clonen.
Ook is Io een dynamische taal, dus objecten kunnen tijdens runtime worden uitgebreid.

Hoewel de syntax van Io heel minimalistisch is, gebeurt er veel achter de schermen - meer dan we in dit project kunnen beschrijven.
Het doel van dit werkstuk is om de standaardtaal While uit te breiden met de belangrijkste mogelijkheden van Io.
Deze uitbreiding noemen we WhIole.
Aangezien het bij Io allemaal draait om objecten en hoe die met elkaar communiceren, zal de kracht van ons project vooral zitten in het uitwerken van die twee concepten.
%Verder laten we zien dat hoewel Io geen keywords, zoals `if' en `for', het nog steeds dezelfde uitdrukkingskracht heeft als While. %Of willen we While en WhileIO vergelijken?
Eerst leggen werken we uit hoe Io werkt en hoe dit verschilt met het door ons gemaakte WhIole. Hierbij zullen we een voorbeeld gebruiken dat in Io geschreven is dat voorbeeld vertalen we naar WhIole. Vervolgens leggen we de syntax en semantiek van WhIole vast door middel van natuurlijke semantiek. Tot slot werken we het hierboven genoemde voorbeeldprogramma uit met de door ons gedefinieerde semantiek. 

\subsection*{Voorbeeld}
Om een idee te krijgen van hoe een programma eruit ziet in Io, volgt hier een voorbeeld:
\begin{lstlisting}[frame=single]
Sheep := Object clone;
Sheep legCount := 4;
MutantSheep := Sheep clone;
MutantSheep legCount = 7;
Dolly := MutantSheep clone;
MutantSheep growMoreLegs := method(n, legCount = legCount + n);
Dolly growMoreLegs(2);
\end{lstlisting}
Dit voorbeeld is afkomstig van \url{andriybuday.com/2012/07/io-programming-language.html} en laat een aantal interessante eigenschappen van Io aan bod komen. Laten we eens kijken wat er in dit voorbeeld gebeurt.

In de eerste regel maken we een kloon van Object. Object is het prototype dat er altijd vanaf het begin al is,
alle objecten die er later bij komen zijn op Object gebaseerd.
Zoals de naam al suggereert maak je, als je een prototype kloont, een nieuwe instantie van dat prototye.
%todo: clone maakt een nieuw (leeg) object aan met als prototype het orginele object
Deze nieuwe instantie heeft precies dezelfde structuur en kenmerken als het prototype waar het uit gekloond is.
Als er later veranderingen worden gemaakt aan het prototype dat gekloond is, dan hebben die ook effect op het nieuwe object.
Sheep is dus een nieuw object dat precies hetzelfde kan en eruit ziet als Object.\newline

Dan willen we zeggen dat het schaap vier poten heeft.
We doen dit door Sheep een nieuw attribuut ``legCount'' te geven en dit attribuut stellen we gelijk aan 4.
Hoewel Sheep een kloon is van Object, zijn het verschillende instanties, en dus krijgt Object er geen attribuut ``legCount'' bij.\newline

Ons schaap is veel te saai. Dit gaan we oplossen door een gemuteerd schaap te maken dat een kloon is van Sheep. Vervolgens zetten we het attribuut ``legCount'' van MutantSheep op 7.\newline

%todo: dolly met kleine letter als instantie van MutantSheep?
In de volgende stap maken we weer een kloon van ons gemuteerde schaap. Deze kloon noemen we dolly. De naam dolly, is naar conventie zonder hoofdletter geschreven.
Dit is om het idee te geven dat het een instantie is van het object MutantSheep.
Dit is alleen om het programmeren overzichtelijker te maken, want Io maakt geen onderscheid tussen klassen en instanties: er zijn alleen maar objecten.\newline

Daarna gaan we een nieuwe methode maken en toevoegen aan MutantSheep, namelijk growMoreLegs.
growMoreLegs neemt één argument n en verhoogt ``legCount'' met die n. 
Zoals eerder gezegd worden veranderingen op een prototype ook overgedragen op prototypes die eruit gekloond zijn. dolly heeft deze methode nu dus ook en kan er gebruik van maken.\newline

Ten slotte willen we dolly nog een paar extra poten geven; twee om precies te zijn.
Om dit te doen maken we gebruik van de nieuwe methode growMoreLegs.
We roepen de methode growMoreLegs van dolly aan met argument 2.
De methode verhoogt het attribuut legCount van dolly met 2.
Uiteindelijk heeft dolly dus 9 poten.

We hebben de code van het bovenstaande voorbeeld zo letterlijk mogelijk vertaald naar WhIole.
Het vertaalde voorbeeld is het volgende:
\begin{lstlisting}[frame=single]
CLONE this AS Sheep;
SET Sheep SLOT legCount TO 4;
CLONE Sheep AS MutantSheep;
SET MutantSheep SLOT legCount TO 7;
CLONE MutantSheep AS Dolly;
SET MutantSheep SLOT growMoreLegs TO SET this SLOT legCount TO x + FROM this CALL legCount;
FROM Dolly CALL growMoreLegs WITH x := 2;
\end{lstlisting}
In het hoofdstuk Uitwerking zullen we laten zien dat de bovenstaande code precies hetzelfde doet als het originele Io programma.

\pagebreak
\section{Syntax}
In het hoofdstuk hierboven staat al een voorbeeldprogramma in WhIole. Maar hoe zit de syntax er precies uit? Dat zullen we in dit hoofdstuk definiëren. Voor WhIole hebben we de standaardtaal While als uitgangspunt genomen. Daarom komt een groot deel van de syntax ook overeen met deze taal.

Voor het definieren van de Syntax maken we gebruik van de volgende sets: 
\begin{enumerate}[noitemsep]
	\item[] $n$ is een cijfer, $Num$.
	\item[] $a$ is een aritmetisch expressie, $Aexp$.
	\item[] $b$ is een booleaanse expressie, $Bexp$. 
	\item[] $S$ is een statement, $Stm$. 
	\item[] $x$ is een variabel, $Var$. 
\end{enumerate}	 

Hierbij kan een variabel naar verschillende typen waarde kan verwijzen, zoals een getal en een object.
De definities voor aritmetische en booleaanse expressies blijven gelijk: 
\begin{gather*}
a ::= n 
	\enspace|\enspace a1 + a2 
	\enspace|\enspace a1 * a2 
	\enspace|\enspace a1 - a2 
	\enspace|\enspace var \\
b ::= true 
	\enspace|\enspace false 
	\enspace|\enspace a1 = a2
	\enspace|\enspace a1 <= a2
	\enspace|\enspace ~b
	\enspace|\enspace b1 /\ b2 
	\enspace|\enspace var 
\end{gather*}
Een groot verschil met While is dat een variabel niet alleen uit de standaard state kunnen komen, maar dat deze ook uit een object gehaald kan worden.
\begin{gather*}
var ::= x
	\enspace|\enspace FROM \enspace var \enspace CALL \enspace x 
\end{gather*}
Onze statements zien er als volgt uit:
\begin{align*}
S ::=& SKIP 	
	\enspace|\enspace  IF\enspace b \enspace THEN \enspace S1 
	\enspace|\enspace  ROM\enspace var\enspace CALL\enspace x_i \enspace Dv
	\enspace|\enspace S1;S2 	
	\enspace|\enspace \\&  CLONE \enspace var_p\enspace AS\enspace var_c 
	\enspace|\enspace SET \enspace var_o \enspace SLOT \enspace x \enspace TO \enspace i  
	\enspace|\enspace WHILE\enspace b\enspace DO\enspace S1 \enspace
\end{align*}
Met de volgende hulpdefinities:	
\begin{gather*}
i ::= a
	\enspace|\enspace b 
	\enspace|\enspace S \\
Dv ::= \epsilon
	\enspace|\enspace WITH \enspace x \enspace := \enspace i; Dv 
\end{gather*}
Hierbij is $\epsilon$ een lege declaratie. $\epsilon$ zelf zul je dus nooit in een programma tegenkomen. 

We hebben dus de volgende drie statements toegevoegd: 
\begin{flalign*}
CLONE 	\enspace var_p	\enspace AS 	\enspace var_c
\end{flalign*} 
Dit maakt een slot in het huidige object met als naam $var_c$. Hierin komt een clone van $var_p$. 

\begin{gather*}
FROM \enspace var\enspace CALL\enspace x_i\enspace Dv \\
Dv 	::= E
	\enspace|\enspace WITH\enspace x\enspace :=\enspace i; Dv
\end{gather*}
Dit levert het resultaat op als je slot $x_i$ van object var aanroept met de variabelen Dv. In een slot kan natuurlijk ook een getal of booleaanse waarde zitten, dus Dv kan ook leeg zijn. 

\begin{align*}
	 SET \enspace var_o\enspace SLOT\enspace x\enspace TO\enspace i
\end{align*} 
Dit maakt in object $var_o$ een slot aan met daarin $i$. $i$ kan zowel een booleaanse expressie, een aritmetische expressie als een methode zijn. 
%todo: constructie woord?
Zoals je kunt zien eindigen niet alle statements in een constructiewoord. Hierdoor is het niet altijd duidelijk wanneer het ene statement is afgelopen en het andere begint. Kijk maar naar:
\begin{align*}
WHILE \enspace x <= 2 \enspace DO \enspace SKIP; \enspace SKIP
\end{align*}
Dit kan op twee manieren geparseerd worden:
\begin{gather*}
WHILE\enspace b\enspace DO\enspace (S1;S2) \\
(WHILE\enspace b\enspace DO\enspace S1); S2. 
\end{gather*} 
Om dit probleem tegen te gaan, nemen wij aan dat je door middel van haakjes statements mag groeperen, net zoals bij While.
\pagebreak

\section{Semantiek}

Als basis gebruiken we de semantiek van While, die we op een paar plekken uitbreiden en aanpassen.
Zo blijven de booleaanse waarden en nummers gelijk aan While:

We definieren nummers als:
\begin{align*}
n ::= 0 \enspace|\enspace 1 \enspace|\enspace n0 \enspace|\enspace n1
\end{align*}

Om deze nummers om te zetten naar echte getallen gebruiken we de volgende functie:
\begin{gather*} 
\mathcal{N}\llbracket 0 \rrbracket = 0  \\
\mathcal{N}\llbracket 1 \rrbracket = 1 \\
\mathcal{N}\llbracket n0 \rrbracket = 2 * \mathcal{N}\llbracket n \rrbracket \\
\mathcal{N}\llbracket n1 \rrbracket = 2 * \mathcal{N}\llbracket n \rrbracket + 1 
\end{gather*}
Ook gebruiken we dezelfde booleaanse waarden. Omdat we deze op veel plekken gebruiken geven we deze set een naam:
\begin{align*}
\mathds{B} = \{\boldsymbol{tt},\boldsymbol{ff}\}
\end{align*}

Onze eigen toevoeging is natuurlijk objecten. Omdat we objecten graag by-reference willen kunnen aanspreken gebruiken we een extra set referentienummers $Ref$ met als meta-variabele $r$. Hierbij zijn referenties unieke waarden. Om deze waarden makkelijk te genereren zien we referenties als natuurlijke getallen. 
\begin{gather*}
r \in \mathcal{N}
\end{gather*}
Om referenties makkelijk op te kunnen schrijven zullen we de volgende afkorting gebruiken:
%todo: juist geen set toch?
\begin{gather*}
r_x = \{x, x \in Ref\}
\end{gather*}
Objecten zelf definiëren we als volgt:
\begin{gather*}\mathcal{O} : Var \rightarrow \mathcal{I} \end{gather*}
Hierbij is Inhoud alles wat je in een object zou kunnen stoppen: 
\begin{gather*}\mathcal{I} = Z \cup \mathds{B} \cup Ref \cup Stm \end{gather*} 

Hierbij gebruiken we dezelfde notatie als bij While om de inhoud van zo'n object weer te geven.
Neem een object o met als waarden x = 4 en y = true. Dit wordt als volgt gerepresenteerd:
\begin{gather*}[x \mapsto 4, y \mapsto tt]\end{gather*}
Om naar objecten te kunnen refereren hebben we ook een referentiefunctie nodig. De signatuur van deze functie is:
\begin{gather*}\mathcal{L} : Ref \rightarrow \mathcal{O}\end{gather*}	


Nu de basistypen gedefinieerd zijn kunnen we met de echte semantiek verder. Om te beginnen passen we de basisovergang van While aan. Deze is normaal gedefinieerd als:
\begin{gather*}
 \langle S,s \rangle \rightarrow s\prime 
\end{gather*}
Waarbij s de state is: Een functie met het type
\begin{gather*}
Var \rightarrow \mathcal{Z}
\end{gather*}
Omdat wij in de syntax geen directe assignment $(x := a)$ meer toestaan, maken wij hier een aanpassing op.\\
We gebruiken nog steeds dezelfde overgang, maar nu met een andere definitie voor de state, waarbij de state equivalent is aan een object. Ook moeten we een lijst met referenties bijhouden. Dit duiden we aan als:
\begin{gather*}
 \langle S,o,l \rangle \rightarrow \langle o\prime, l\prime \rangle 
\end{gather*}
Daarnaast moeten wij nog definiëren dat aan de start van een programma de state niet leeg is, maar de varnaam $ this $ bind aan een leeg object $ [\enspace] $.
\begin{gather*}
o_{start} = [this \rightarrow r_0]	
l_{start} = [r_0 \mapsto [ \enspace ]]
\end{gather*}
Daarnaast moeten we enkele veranderingen maken aan de manier waarop aritmetische en booleaanse expressies worden geëvalueerd. 
Allereerst is het belangrijk om op de juiste manier een stuk inhoud op te halen. We definiëren een totale functie $GET$ die een syntactisch construct (De varnaam), een gewenst type en een object neemt. De functionaliteit leggen we vast als:
\[ GET: (x,t) \rightarrow (\mathcal{O} \rightarrow \mathcal{I}) \]
\[
\begin{matrix}
GET\llbracket (x,t) \rrbracket_o & = & 
\begin{cases}
\begin{matrix}
o(x) & \mbox{ als } x \in domein(o) \mbox{ en } o(x) \in t\\
GET\llbracket x,t \rrbracket_{o(proto)} & anders
\end{matrix}
\end{cases}\\
\end{matrix}
\]

Na dit te hebben gedefinieerd kunnen we met behulp van de GET-functie enkele nieuwe functies definiëren, waarmee we syntactische constructen om kunnen zetten naar getallen en booleaanse variabelen. We definiëren de totale functie $ \RA $ die een variabel, een object en een referentielijst neemt. De functionaliteit leggen we vast als volgt:

\[ \RA : Aexp \rightarrow ((\mathcal{O}, \mathcal{L}) \rightarrow \mathds{Z} ) \]
\[
\begin{matrix}
\RA\llbracket n \rrbracket_{(o,l)} & = & \mathcal{N}\llbracket n \rrbracket\\
\RA\llbracket x \rrbracket_{(o,l)} & = & GET\llbracket (x,\mathds{Z}) \rrbracket_{o}\\
\RA\llbracket FROM \enspace var \enspace CALL \enspace x \rrbracket_{(o,l)} & = & GET\llbracket (x,\mathds{Z}) \rrbracket_{\RO \llbracket GET\llbracket (var,Ref) \rrbracket_o \rrbracket_l }\\
\RA\llbracket a_1 + a_2 \rrbracket_{(o,l)} & = & \RA\llbracket a_1 \rrbracket_{(o,l)} + \RA\llbracket a_2 \rrbracket_{(o,l)}\\
\RA\llbracket a_1 * a_2 \rrbracket_{(o,l)} & = & \RA\llbracket a_1 \rrbracket_{(o,l)} * \RA\llbracket a_2 \rrbracket_{(o,l)}\\
\RA\llbracket a_1 - a_2 \rrbracket_{(o,l)}& = & \RA\llbracket a_1 \rrbracket_{(o,l)} - \RA\llbracket a_2 \rrbracket_{(o,l)}\\
\end{matrix}
\]

We definiëren ook eenzelfde totale functie $ \RB $:
\[ \RB : Bexp \rightarrow ((\mathcal{O},\mathcal{L}) \rightarrow \mathds{B} ) \]
\[
\begin{matrix}
\RB \llbracket true \rrbracket_{(o,l)} & = & \boldsymbol{tt}\\
\RB \llbracket false \rrbracket_{(o,l)} & = & \boldsymbol{ff}\\

\RB \llbracket x \rrbracket_{(o,l)} & = & GET\llbracket (x,\mathds{B}) \rrbracket_o \\

\RB \llbracket FROM \enspace var \enspace CALL \enspace x \rrbracket_{(o,l)} & = & GET\llbracket (x,\mathds{B}) \rrbracket_{\RO \llbracket GET\llbracket (var,Ref) \rrbracket_o \rrbracket_l}\\

\RB \llbracket a_1 = a_2 \rrbracket_{(o,l)} & = &
\begin{cases}
\begin{matrix}
\boldsymbol{tt} & als \enspace \RA\llbracket a_1 \rrbracket_{(o,l)} = \RA\llbracket a_2 \rrbracket_{(o,l)} \\
\boldsymbol{ff} & als \enspace \RA\llbracket a_1 \rrbracket_{(o,l)} \not= \RA\llbracket a_2 \rrbracket_{(o,l)}
\end{matrix}
\end{cases}\\

\RB \llbracket a_1 \leq a_2 \rrbracket_{(o,l)} & = &
\begin{cases}
\begin{matrix}
\boldsymbol{tt} & als \enspace \RA\llbracket a_1 \rrbracket_{(o,l)} \leq \RA\llbracket a_2 \rrbracket_{(o,l)} \\
\boldsymbol{ff} & als \enspace \RA\llbracket a_1 \rrbracket_{(o,l)} > \RA\llbracket a_2 \rrbracket_{(o,l)}
\end{matrix}
\end{cases}\\

\RB \llbracket \neg b \rrbracket_{(o,l)} & = &
\begin{cases}
\begin{matrix}
\boldsymbol{tt} & als \enspace \RB\llbracket b \rrbracket_{(o,l)} = \boldsymbol{ff} \\
\boldsymbol{ff} & als \enspace \RB\llbracket b \rrbracket_{(o,l)} = \boldsymbol{tt}
\end{matrix}
\end{cases}\\

\RB \llbracket b_1 \wedge b_2 \rrbracket_{(o,l)} & = &
\begin{cases}
\begin{matrix}
\boldsymbol{tt} & als \enspace \RB\llbracket b_1 \rrbracket_{(o,l)} = \boldsymbol{tt} en \RB \llbracket b_2 \rrbracket_{(o,l)} = \boldsymbol{tt}\\
\boldsymbol{ff} & als \enspace \RB\llbracket b_1 \rrbracket_{(o,l)} = \boldsymbol{ff} of \RB \llbracket b_2 \rrbracket_{(o,l)} = \boldsymbol{ff}
\end{matrix}
\end{cases}\\

\end{matrix}
\]

En als laatste eenzelfde functie die een referentie aanneemt, en deze omzet in een object.
\begin{align*} 
\RO& : Ref \rightarrow ( \mathcal{L} \rightarrow \mathcal{O}) \\
\RO& \llbracket r \rrbracket_l  =  l(r)
\end{align*}

Semantiekregels:\\
\renewcommand*{\arraystretch}{2.5}
\[
\begin{matrix}
[skip_{ns}]  & \langle skip, o, l \rangle \rightarrow \langle o, l \rangle\\
[comp_{ns}]  & \dfrac{\langle S1, o, l \rangle \rightarrow \langle o', l' \rangle \langle S2, o', l' \rangle \rightarrow \langle o'', l'' \rangle}{\langle S1;S2, o, l \rangle \rightarrow \langle o'', l'' \rangle}\\
[if_{ns}^{tt}] & \dfrac{\langle S1, o, l \rangle \rightarrow \langle o', l' \rangle}{\langle \mbox{IF b THEN S1 ELSE S2}, o, l \rangle \rightarrow \langle o', l' \rangle} &&\mbox{ als }\RB \llbracket b \rrbracket_{(o,l)} = \boldsymbol{tt}\\
[if_{ns}^{ff}] & \dfrac{\langle S2, o, l \rangle \rightarrow \langle o', l' \rangle}{\langle \mbox{IF b THEN S1 ELSE S2}, o, l \rangle \rightarrow \langle o', l' \rangle} &&\mbox{ als }\RB \llbracket b \rrbracket_{(o,l)} = \boldsymbol{ff}\\
[while_{ns}^{tt}] & \dfrac{\langle S, o, l \rangle \rightarrow \langle o', l' \rangle \langle \mbox{WHILE b DO S}, \langle o', l' \rangle \rightarrow \langle o'',l'' \rangle}{\mbox{WHILE b DO S}, o, l \rangle \rightarrow \langle o'',l'' \rangle} &&\mbox{ als }\RB \llbracket b \rrbracket_{(o,l)} = \boldsymbol{tt}\\
[while_{ns}^{ff}] & \langle \mbox{WHILE b DO S}, o, l \rangle \rightarrow \langle o,l \rangle &&\mbox{ als }\RB \llbracket b \rrbracket_{(o,l)} = \boldsymbol{ff}\\
\end{matrix}
\]
%$
%\begin{matrix}
%S
%[clone_{ns}] & \langle CLONE \enspace var_p \enspace AS \enspace var_c, o, l> %\rightarrow \langle o\prime, l\prime> \\ 
%\end{matrix}
%$
%\begin{align*}
%	waar: \\
%	o\prime &= o[var_c \mapsto r_{|l|+1}] &&\\
%	l\prime &= l[ref_{this} \mapsto this[var_c \mapsto r_{|l|+1}] , r_{|l|+1} \mapsto [ proto \mapsto ref_p]]&&\\
%	ref_p &= GET \llbracket var_p, Ref \rrbracket_o &&\\
%	ref_{this} &= GET \llbracket this, Ref \rrbracket_o &&\\
%	this &= GET \llbracket this, Ref \rrbracket_o &&\\	
%\end{align*}

\[
\begin{matrix}

[clone_{ns}] &
\begin{matrix}
\langle CLONE \enspace var_p \enspace AS \enspace var_c, o, l\rangle 
\rightarrow\\
\langle o[var_c \mapsto r_{|l|}], l[GET \llbracket this, Ref \rrbracket_o \mapsto \RO \llbracket GET \llbracket this, Ref \rrbracket_o \rrbracket_l[var_c \mapsto r_{|l|}] ,\\ r_{|l|} \mapsto [ proto \mapsto GET \llbracket var_p, Ref \rrbracket_o]]\rangle
\end{matrix}\\[60pt]
%\hline
\begin{matrix}
[set_{ns}^{a}]\\
\mbox{ als } i \in Aexp\\
\end{matrix} &
\begin{matrix}
\langle SET \enspace var_t \enspace SLOT \enspace x \enspace TO \enspace i, o, l \rangle\\
\rightarrow
\langle o, l[GET \llbracket var_t, Ref \rrbracket_o \mapsto \RO \llbracket GET \llbracket var_t, Ref \rrbracket_o \rrbracket_l[x \mapsto \RA 
\llbracket i \rrbracket_{(o,l)}]] \rangle
\end{matrix}\\[60pt]
%\hline
\begin{matrix}
[set_{ns}^{b}]\\
\mbox{ als } i \in Bexp\\
\end{matrix} &
\begin{matrix}
\langle SET \enspace var_t \enspace SLOT \enspace x \enspace TO \enspace i, o, l \rangle\\
\rightarrow
\langle o, l[GET \llbracket var_t, Ref \rrbracket_o \mapsto \RO \llbracket GET \llbracket var_t, Ref \rrbracket_o \rrbracket_l[x \mapsto \RB 
\llbracket i \rrbracket_{(o,l)}]] \rangle
\end{matrix}\\[60pt]
%\hline
\begin{matrix}
[set_{ns}^{a,not}]\\
\mbox{ als } i \notin Aexp \cup Bexp\\
\end{matrix} &
\begin{matrix}
\langle SET \enspace var_t \enspace SLOT \enspace x \enspace TO \enspace i, o, l \rangle\\
\rightarrow
\langle o, l[GET \llbracket var_t, Ref \rrbracket_o \mapsto \RO \llbracket GET \llbracket var_t, Ref \rrbracket_o \rrbracket_l[x \mapsto i ]] \rangle
\end{matrix}\\
\end{matrix}\\
\]
\[
\begin{matrix}
[call_{ns}] &
\dfrac
{
	\begin{matrix}
	\langle Dv', o, l \rangle \rightarrow \langle o\prime, l\prime \rangle\\
	\langle GET \llbracket x, \mathcal{I} \rrbracket_{\RO \llbracket GET\llbracket var, \mathcal{O} \rrbracket_o \rrbracket_l}, o\prime [proto \mapsto GET\llbracket var, \mathcal{O} \rrbracket_o, this \mapsto GET\llbracket var \rrbracket_o], l\prime \rangle\\
	\rightarrow
	\langle o\prime\prime, l\prime\prime \rangle
	\end{matrix}	
}
{
	\langle FROM \enspace var \enspace CALL \enspace x \enspace Dv,o,l \rangle
	\rightarrow
	\langle o, l\prime\prime \rangle
} \\[60pt]
%\hline
[DvSkip_{ns}] &
\langle \epsilon ,o,l \rangle \rightarrow \langle [],l \rangle \\[60pt]
%\hline
\begin{matrix}
[DvVar_{ns}^{a}]\\
\mbox{ als } i \in Aexp 
\end{matrix}
 &
\dfrac
{	\langle Dv,o,l \rangle \rightarrow \langle o\prime, l \rangle }
{
	\langle WITH \enspace x \enspace := \enspace i; \enspace Dv, o, l \rangle
	\rightarrow
	\langle o\prime[x \mapsto \RA \llbracket i \rrbracket_o], l \rangle
}\\[60pt]
%\hline
\begin{matrix}
[DvVar_{ns}^{b}]\\
\mbox{ als } i \in Bexp
\end{matrix}
&
\dfrac
{	\langle Dv,o,l \rangle \rightarrow \langle o\prime, l \rangle }
{
	\langle WITH \enspace x \enspace := \enspace i; \enspace Dv, o, l \rangle
	\rightarrow
	\langle o\prime[x \mapsto \RB \llbracket i \rrbracket_o], l \rangle
}\\[60pt]
%\hline
\begin{matrix}
[DvVar_{ns}^{S}]\\
\mbox{ als } i \notin Aexp \cup Bexp
\end{matrix}
&
\dfrac
{	\langle Dv,o,l \rangle \rightarrow \langle o\prime, l \rangle }
{
	\langle WITH \enspace x \enspace := \enspace i; \enspace Dv, o, l \rangle
	\rightarrow
	\langle o\prime[x \mapsto i ], l \rangle
}\\

\end{matrix}
\]

%\end{matrix}
%\]

\pagebreak
\section{Uitwerking}
Hier volgt nogmaals het naar WhIole vertaalde voorbeeld programma.

\begin{lstlisting}[frame=single]
CLONE this AS Sheep;
SET Sheep SLOT legCount TO 4;
CLONE Sheep AS MutantSheep;
SET MutantSheep SLOT legCount TO 7;
CLONE MutantSheep AS Dolly;
SET MutantSheep SLOT growMoreLegs TO SET this SLOT legCount TO x + FROM this CALL legCount;
FROM Dolly CALL growMoreLegs WITH x := 2;
\end{lstlisting}

Dit programma werken we de ``boom'' uit te schrijven.
Deze boom is echter veel te groot om op één blaadje te passen.
Alle transities die in de boom voorkomen staan genummerd in de bijlage boom.txt.
Deze nummers zijn in onderstaande boom gezet om de structuur aan te geven.

NIET AF!!!!!!
%$
%\dfrac
%	{1 \dfrac{\dfrac{3 \dfrac{5 \dfrac{7 \dfrac{9}{\mbox{8 10}}{8}}{6}}{4}}{2}} {0}}
%$

Volgens de boom heeft Dolly uiteindelijk negen poten, net zoals in het voolbeeld met Io.
Onze semantiek lijkt dus te werken.
\iffalse
\appendix
\section{Planning}
Vrijdag 29 april: \\Het werkstuk voldoet aan de eisen van eerste versie.\newline\newline %hebben we een alternatief voor 2x \newline?
Dinsdag 10 mei: \\Het hoofdstuk over de syntax van Io is helemaal klaar en willen we al aardig op weg zijn met de semantiek.\newline\newline
Dinsdag 24 mei: \\Het hoofdstuk over de semantiek is compleet. We bedoelen hier niet mee dat dat hoofdstuk helemaal klaar en perfect is, maar
dat het zodanig is dat we het kunnen gebruiken om de rest van het werkstuk te kunnen maken.\newline\newline
Vrijdag 27 mei: \\Het werkstuk voldoet aan de eisen van de tweede versie. Dat wil zeggen: het inleidende hoofdstuk is af, het hoofdstuk over de syntax is af, het hoofdstuk over de semantiek is af, het hoofdstuk over de analyse bevat in elk geval het programma dat geanalyseerd gaat worden.\newline\newline
Vrijdag 17 juni: \\De definitieve versie van het werkstuk is gereed en ingeleverd.
\fi
\end{document}  
