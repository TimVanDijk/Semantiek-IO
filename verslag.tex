\title{Semantiek van Io}
\author{
          Tim van Dijk
          S4477073
        \and
          Nikki van der Gouw
          S4463412
        \and
          Robin Tonen 
          S4486668
}
\date{\today}

\documentclass[12pt]{article}
\usepackage{hyperref}
\usepackage{graphicx}
\usepackage[dutch]{babel}
\usepackage[bottom]{footmisc}
\graphicspath{{images/}}

\begin{document}
\maketitle

\begin{abstract}
\end{abstract}

\section{Introductie Io}
Zorg dat je een korte inleiding schrijft waarin je zo concreet mogelijk uitlegt wat je gaat doen. Beschrijf hierbij ook wat er zo bijzonder is aan het gekozen onderwerp ten opzichte van de standaardtaal While. Neem in de inleiding ook alvast een voorbeeldprogramma op waarin iets interessants gebeurt. Zo'n programma helpt om duidelijk te krijgen hoe programma's in jullie taal er uitzien.
\subsection{Doel}
\subsection{Karakteristieken}
\subsection{Voorbeeld}

\section{Syntax}
Zorg dat je een beschrijving geeft van de syntax die je gaat gebruiken. Dat kan natuurlijk al een complete grammatica zijn, maar op dit moment is het voldoende om hier te beschrijven welke constructies uitgewerkt gaan worden. In het bijzonder geef je hier aan of je de hele taal gaat beschrijven of slechts een deel er van. Verwacht je nu al problemen bij de beschrijving van de syntax, benoem die hier dan al.

\section{Semantiek}
Zorg dat je een beschrijving geeft van de semantiek die je gaat gebruiken. Ga je voor ns, sos of nog heel iets anders? En waarom? Probeer ook vast iets te zeggen over de concepten die je nodig hebt: wat zijn je toestanden (als je toestanden gebruikt), wat zijn je transities, welke types spelen een rol, etcetera. Natuurlijk hoef je hier nog geen complete lijst met semantiekregels te geven, maar je moet al wel een idee hebben hoe je denkt te gaan werken. Verwacht je nu al problemen bij de beschrijving van de semantiek, benoem die hier dan al.
\subsection{Messages}
\subsection{Objecten}

\section{Analyse}
Zorg dat je kort beschrijft wat voor een analyse je gaat doen. Heb je bijvoorbeeld al een mooi stuk voorbeeldcode waarvan je uiteindelijk wil laten zien dat het precies doet wat in je semantiekregels hebt vastgelegd, geef dat voorbeeld dan reeds expliciet aan. Dit mag natuurlijk het voorbeeld uit de inleiding zijn, maar het hoeft niet.

\section{Conclusie}

\appendix
\section{Planning}
Zorg dat je in een appendix een globale planning opneemt. Hierbij is het natuurlijk niet de bedoeling dat je alleen maar opschrijft wanneer de deadlines zijn uit het weekoverzicht, maar het gaat er juist om dat je aangeeft op welke datum je welk deel van het werkstuk af hebt. Wanneer is het hoofdstuk over syntax af? Wanneer liggen de concepten van de semantiek vast? Wanneer liggen de regels van de semantiek vast? Enzovoorts. In het uiteindelijke werkstuk is die planning natuurlijk verdwenen, want dan is het werkstuk klaar...

\end{document}  